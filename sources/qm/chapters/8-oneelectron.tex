\documentclass[../qm.tex]{subfiles}
\begin{document}
\section{Interaction of Particles with EM Fields}
	Before diving into a full computation of an electromagnetic Hamiltonian, we need to properly ``quantize'' it. Having $\vecopr{A}$ our vector potential operator and $\opr{\phi}$ our scalar potential operator, and knowing that $\vec{P}=\vec{p}-q\vec{A}$, where $\vec{P}$ is the minimal coupling momentum of an EM field, we have that the classical Hamiltonian for an electromagnetic field is
	\begin{equation}
		\ham=\frac{1}{2m}\left( p_i-qA_i \right)^2+q\phi
		\label{eq:classemham}
	\end{equation}
	Quantizing, and noting that $\comm{\opr{p}_i}{\opr{A}_i}\ne0$ generally, we have, setting the Coulombe gauge ($\partial_iA^i=0,\ \phi=0$) that the Hamiltonian in study for our quantum system is the following
	\begin{equation}
		\opr{\ham}=\frac{1}{2m}\left( -\hbar^2\nabla^2+q^2\opr{A}^2+2iq\hbar\vecopr{A}\cdot\nabla\right)
		\label{eq:qhamemfield}
	\end{equation}
	The interaction of the field with the particle can be analyzed perturbatively, and considering first order terms in $\vecopr{A}$ we have
%	\begin{equation}
%		\opr{\ham}=-\frac{\hbar^2}{2mr^2}\pdv{r}\left( r^2\pdv{r} \right)+\frac{\opr{L}^2}{2mr^2}-\frac{e^2}{r}+\frac{q^2}{2m}\opr{A}^2+\frac{iq\hbar}{m}\vecopr{A}\cdot\nabla
%		\label{eq:fullpert}
%	\end{equation}
%	We already know that for weak fields the term with $A^2$ is negligible, and our Hamiltonian for weak fields is
	\begin{equation}
		\opr{\ham}=\opr{\ham}_0+\frac{iq\hbar}{m}\vecopr{A}\cdot\nabla
		\label{eq:weakfieldemham}
	\end{equation}
	Where we put $\opr{\ham}_0$ as the unperturbed Hamiltonian.\\
	Calling the perturbative piece of the Hamiltonian $\opr{W}(t)$ we have that the perturbation on a transition between a state $\ket{a}$ and $\ket{b}$ will be given by the following integral
	\begin{equation}
		W_{ba}(t)=\frac{q}{m}\bra{b}\vecopr{A}\cdot\vecopr{p}\ket{a}
		\label{eq:empertmatrix}
	\end{equation}
	Where the vector potential is the solution of the following integral
	\begin{equation}
		\hat{A}_i(x_i,t)=\frac{1}{2}\int_{0}^{\infty}A_0(\omega)\epsilon_i\left( e^{i(k_ix^i-\omega t+\delta_{\omega})}+e^{-i(k_ix^i-\omega t +\delta_{\omega})} \right)\diff{\omega}
		\label{eq:vectorpotentialem}
	\end{equation}
	Take now a time evolved eigenstate of the unperturbed Hamiltonian, and rewrite as a linear combination of those times an unknown time-dependent function $c_k(t)$. We can write, for a generic eigenstate $\ket{k}$, that the new perturbed state will be $\ket{\psi}$
	\begin{equation}
		\ket{\psi}=\sum_kc_k(t)e^{-\frac{iE_kt}{\hbar}}\ket{k}
		\label{eq:perturbedempsi}
	\end{equation}
	Inserting everything in the time-dependent Schrödinger equation for the perturbed Hamiltonian, and simplifying terms, we have
	\begin{equation}
		i\hbar\sum_k\dot{c}_k(t)e^{-\frac{iE_kt}{\hbar}}\ket{k}=\sum_kc_k\hat{W}(t)e^{-\frac{iE_kt}{\hbar}}\ket{k}
		\label{eq:tdsep}
	\end{equation}
	Checking now the transition $\ket{k}\to\ket{b}$, and remembering the orthonormality between states, we have, putting $\omega_{bk}=(E_b-E_k)/\hbar$
	\begin{equation}
		\dot{c}_b(t)=\frac{1}{i\hbar}\sum_kc_k(t)\bra{b}\hat{W}\ket{k}e^{-i\omega_{bk}t}
	\end{equation}
	Approximating $c_k(t)$ to the first order in $\lambda$ and making the assumption that the system will be in a state $\ket{a}$ for $t\le0$, therefore implying $c_k(t)=\delta_{ka}$ for $k\ne a,b$, we have, rearranging terms of the same order
	\begin{equation}
		\dot{c}_b(t)=\frac{1}{i\hbar}\bra{b}\hat{W}\ket{a}e^{-i\omega_{ba}t}
	\end{equation}
	Factoring out constants and time dependent parameters, we have that the coefficient for our perturbation is
	\begin{equation}
		c_b(t)=\frac{q}{2im\hbar}\int_{0}^{\infty}\int_0^tA_0(\omega)\bra{b}e^{ik^ix_i}\epsilon\cdot\vecopr{p}\ket{a}e^{-i(\omega_{ba}-\omega)t'}e^{i\delta_\omega}\diff{t}\dd\omega+c.c.
		\label{eq:transitionprobqem}
	\end{equation}
	If $t>>2\pi/\omega$ we can approximate the radiation as a plane wave solution. Parting the integrals we have that one will be nonzero for $\omega_{ba}\simeq\omega$ and the second for $\omega_{ba}\simeq-\omega$. The first nonzero solution describes an absorption between the state with $\ket{a}$ and $\ket{b}$, with $E_b>E_a$, and the second instead describes an emission between the state $\ket{a}$ and $\ket{b}$.\\
	Now, indicating $\bra{b}e^{ik_ix^i}\epsilon\cdot\vecopr{p}\ket{a}$ as $M_{ba}$ and noting that $\expval{e^{i\delta_{\omega}}}=\delta(\omega)$ we have that the transition probability will be
	\begin{equation}
		\abs{c_b(t)}^2=\frac{q^2}{4m^2\hbar^2}\int_{0}^\infty\abs{A_0(\omega)}^2\abs{\bra{b}e^{ik^ix_i}\vec{\epsilon}\cdot\vecopr{p}\ket{a}}^2F(t,\omega_{ba}-\omega)\dd\omega
		\label{eq:transprob}
	\end{equation}
	Where the function $F$ is defined as the square modulus of the integral of the exponential, which corresponds to a $\mathrm{sinc^2}$ function
	For $t>>1$ we have that, for the properties of the function $F$, Fermi's Golden Rule, and the connection between $A^2$ and the intensity $I$, we have
	\begin{equation}
		P_{a\to b}(t)=\abs{c_b(t)}^2=\frac{\pi q^2t}{2\hbar^2\epsilon_0\omega^2m^2c}I(\omega)\abs{M_{ba}}^2\delta(\omega_{ba}-\omega)
		\label{eq:transprobfermi}
	\end{equation}
	Note that this probability is linearly dependent on time, therefore, the time-weighted probability of transition between the states $\ket{a}$ and $\ket{b}$ for a system in an electromagnetic field is
	\begin{equation}
		W_{ba}=\frac{\pi q^2}{\epsilon_0\omega_{ba}^2m^2c\hbar^2}I(\omega_{ba})\abs{M_{ba}}^2
		\label{eq:perttransem}
	\end{equation}
	\section{Dipole Approximation}
	In order to evaluate dipole transitions we approximate our plane wave solution to the first order, getting $e^{ik_ix^i}\approx1$, which gives us that $M_{ba}=\bra{b}\ver{\epsilon}\cdot\vecopr{p}\ket{a}$. In this approximation, we can write
	\begin{equation*}
		\vecopr{p}=\frac{m}{i\hbar}\comm{\vecopr{r}}{\opr{\ham}_0}
	\end{equation*}
	Substituting in $M_{ba}$ and using $\opr{\ham}_0$'s hermiticity, we have, defining an \textit{electric dipole operator} $\vecopr{D}=-e\vecopr{r}$
	\begin{equation}
		M_{ba}^{(1)}=\frac{im\omega_{ba}}{e}\bra{b}\ver{\epsilon}\cdot\vecopr{D}\ket{a}
		\label{eq:mbadipole}
	\end{equation}
	And
	\begin{equation}
		W_{ba}^{(1)}=\frac{\pi I(\omega_{ba})}{\hbar^2c\epsilon_0}\abs{\bra{b}\ver{\epsilon}\cdot\vecopr{D}\ket{a}}^2
		\label{eq:transprobdipole}
	\end{equation}
	Evaluating the square norm on the right, we have that in case of unpolarized light, the vector $\epsilon$ lays randomly on a sphere, and therefore we have that the matrix elements of $M_{ba}$ will depend solely on $\ket{b}\vecopr{r}\ket{a}$, as follows
	\begin{equation}
		\abs{\bra{b}\ver{\epsilon}\cdot\vec{r}\ket{a}}^2=\frac{1}{3}\sum_k\abs{\bra{b}\opr{r}_k\ket{a}}^2=\frac{1}{3}\abs{\vec{r}_{ba}}^2
		\label{eq:mbadipoletransunpol}
	\end{equation}
	Where we used that $\expval{\cos^2(\theta)}=1/3$.\\
	Finally, we have that in the dipole approximation
	\begin{equation}
		W_{ba}=\frac{\pi q^2I(\omega_{ba})}{3\hbar^2\epsilon_0c}\abs{\bra{b}\vecopr{r}\ket{a}}^2
		\label{eq:wbadipole}
	\end{equation}
	\subsection{Dipole Selection Rules for Atomic Transitions}
	Consider now a Hydrogenic Hamiltonian (without spin) as our unperturbed system. In this case we choose $q=-e$.\\
	As we have seen before, the electric dipole transition probability between two states with $E_b>E_a$ depends only on the matrix elements of $\ver{\epsilon}\cdot\vec{r}$. Rewriting everything in terms of spherical components, we have
	\begin{equation}
		\begin{aligned}
			\opr{r}_{\pm1}&=\frac{\opr{x}\pm i\opr{y}}{\sqrt{2}},\quad\opr{r}_0=\opr{z}\\
			\ver{\epsilon_{\pm1}}&=\frac{\ver{\epsilon_x}\pm i\ver{\epsilon_y}}{\sqrt{2}},\quad\ver{\epsilon_0}=\ver{\epsilon_z}
		\end{aligned}
		\label{eq:sphcoord}
	\end{equation}
	Considering that Spherical Harmonics are eigenfunctions of our unperturbed Hamiltonian and can also be used to describe the components of the vectors, we have $r_{\pm1}=\sqrt{\frac{4\pi}{3}}Y_{1,\pm1}\opr{r}$ and $\opr{r}_0=\sqrt{\frac{4\pi}{3}}Y_{1,0}\opr{r}$, hence we get that $\ver{\epsilon}\cdot\vec{r}_{ba}$ is
	\begin{equation}
		\sum_q\bra{n'l'm'}\cc{\ver{\epsilon_q}}\opr{r}_q\ket{nlm}=\sum_q\cc{\ver{\epsilon_q}}\sqrt{\frac{4\pi}{3}}\iint\cc{R}_{n'l'}R_{nl}r^3\cc{Y}^{m'}_{l'}Y_{1}^qY^m_l\diff{\Omega}\diff{r}
		\label{eq:selrulesdippolarization}
	\end{equation}
	The last integral is nonzero if and only if $m'=m+q$, hence we must have that $m-m'=\pm1$ and $q=\pm1$. On the other hand, due to the parity of spherical harmonics, we need that $l+l'+1$ must be even, hence summing the coefficients using Clebsch-Gordan rules, we need that $l=l'\pm1$, and hence $l-l'=\pm1$. In bra-ket notation this reduces to the following calculation
	\begin{equation}
		\int\cc{Y}_{l'}^{m'}Y^q_1Y^m_l\diff{\Omega}\to\sqrt{\frac{3(2l+1)}{4\pi(2l'+1)}}\bra{l100}\ket{l'0}\bra{l1mq}\ket{l'm'}
		\label{eq:cgtableselrule}
	\end{equation}
	Which brings back our previously found selection rules, plus one more constraint on $\Delta m$. Recapping everything, we get that for a spinless system, the selection rules for a dipole transition are
	\begin{table}[H]
		\centering
		\begin{tabular}{|c|c|}
			\hline
			Quantum Number&Permitted Transitions\\
			\hline
			$l$&$\pm1$\\
			\hline
			$m$&$0,\pm1$\\
			\hline
		\end{tabular}
		\label{tab:selectionrulesspinless}
	\end{table}
	\subsection{Spontaneous Emission}
	A situation that can't be evaluated using semiclassical methods is that of spontaneous emission, since an atom can decay spontaneously, even without having some radiation stimulating the process.\\
	We start by evaluating an atom-photon system, for which we have three probabilities of interaction between the two: absorption ($B_a$), emission ($B_e$) and spontaneous emission ($A_{se}$). The $B$s indicate the probability in unit time that the photon induces a transition between the two states $b$ and $a$ with $E_b>E_a$, and $A$ indicates the probability for unit time that the state $b$ decays spontaneously to the state $a$. We already know that the $B$s are tied to our previous perturbation matrix as follows
	\begin{equation*}
		W_{ba}=B_{ba}\rho(\omega_{ba})
	\end{equation*}
	Where $\rho$ is a density tied to the photon number, which has the following formal expression
	\begin{equation*}
		\rho(\omega)=\frac{\hbar\omega N(\omega)}{V}
	\end{equation*}
	With $V$ the volume considered and $N(\omega)$ the number of photons in that given frequency.\\
	Given this, the number of atoms that decay from $a$ to $b$ is the following
	\begin{equation*}
		N_{ba}=N_aB_a\rho(\omega_{ba})
	\end{equation*}
	And, vice-versa
	\begin{equation*}
		N_{ab}=N_bB_e\rho(\omega_{ba})+N_bA_{se}
	\end{equation*}
	If the system is at equilibrium we must have $N_{ab}=N_{ba}$, hence
	\begin{equation}
		\frac{N_b}{N_a}=\frac{B_a\rho(\omega_{ba})}{B_e\rho(\omega_{ba})+A_{se}}=e^{-\beta\hbar\omega_{ba}}
		\label{eq:equilibriumse}
	\end{equation}
	Where $\beta=(k_BT)^{-1}$. Knowing that $\rho$ must follow a Black-Body law, we have, solving for $\rho$
	\begin{equation}
		\rho(\omega_{ba})=\frac{A_{se}}{B_a\left( e^{\beta\hbar\omega_{ba}}-\frac{B_e}{B_a} \right)}
		\label{eq:transitiondensityse}
	\end{equation}
	Hence
	\begin{equation}
		\begin{aligned}
			\frac{B_e}{B_a}&=1\\
			\frac{A_{se}}{B_a}&=\frac{\hbar\omega_{ba}^3}{\pi^2c^3}
		\end{aligned}
		\label{eq:coefficients}
	\end{equation}
	We then must have that $W_{ba}^a=W_{ba}^{se}=B_a\rho(\omega_{ba})$, and coupling it to equation \eqref{eq:perttransem}, we must have that
	\begin{equation}
		W_{ba}^{se}=\frac{\pi\omega_{ba}^3e^2}{\hbar c^3\epsilon_0}\abs{\ver{epsilon}\cdot\vecopr{r}_{ba}}^2
		\label{eq:spontaneousemissiontransmatrix}
	\end{equation}
	\subsection{Thermodynamic Equilibrium}
	We already saw how the density of photons at a given frequency is given by the following expression, without proof.
	\begin{equation*}
		\rho(\omega)=\frac{\hbar\omega^3}{\pi^2c^3}\frac{1}{e^{\beta\hbar\omega}-1}
	\end{equation*}
	\begin{proof}
		We need to prove this relation. We start supposing that we have a system of levels with $\Delta E=\hbar\omega$ for every level in the set. The number of photons that populate this system at a temperature $T$ is
		\begin{equation}
			N(\omega,T)=\frac{\sum_{n=0}^{\infty}ne^{-\beta n\hbar\omega}-\frac{\beta\hbar\omega}{2}}{\sum_{n=0}^{\infty}e^{-\beta n\hbar\omega-\frac{\beta\hbar\omega}{2}}}=\frac{\sum_{n=0}^{\infty}nx^n}{\sum_{n=0}^{\infty}x^n}=-x\derivative{\log(1-x)}{x}=\frac{1}{e^{\beta\hbar\omega}-1}
			\label{eq:numberofphotonsaeq}
		\end{equation}
		Where we put, for convenience, $x=e^{-\beta\hbar\omega}$\\
		We now need to evaluate how many states we have with frequency $\omega$. We begin considering a box with sides of length $L$ with periodic boundary value conditions. The number of possible modes is
		\begin{equation*}
			N_m(k)=2\frac{\frac{4}{3}\pi k^3}{\left( \frac{2\pi}{L} \right)^3}
		\end{equation*}
		Differentiating and passing to frequency ($\omega=ck$), we get
		\begin{equation}
			\diff{N_m}(\omega)=\frac{V\omega^2}{\pi^2c^3}\diff{\omega}
			\label{eq:frequencymodes}
		\end{equation}
		And, henceforth, the energy density is
		\begin{equation}
			\rho(\omega)\diff{\omega}=\frac{\hbar\omega}{V}N(\omega,T)\diff{N_m}(\omega)=\frac{\hbar\omega^3}{\pi^2c^3}\frac{1}{e^{\beta\hbar\omega}-1}\diff{\omega}
			\label{eq:rhoomegamodesproof}
		\end{equation}
	\end{proof}
	\section{Relativistic Corrections and Fine Structure}
	Taking the full-blown relativistically invariant Dirac Hamiltonian and considering it in the limit $v/c<<1$ and assuming infinite nuclear mass, we get the following expression
	\begin{equation}
		\opr{\ham}=mc^2+\frac{\opr{p}^2}{2m}-\frac{\opr{p}^4}{8m^3c^2}+\frac{1}{2m^2c^2}\frac{1}{R}\derivative{V}{R}\vecopr{L}\cdot\vecopr{S}+\frac{\hbar^2}{8m^2c^2}\nabla^2V(R)+V(R)
		\label{eq:dirachamiltonianslowspeed}
	\end{equation}
	This Hamiltonian can be divided in three parts: The non-relativistic Hamiltonian, a kinetic relativistic correction, spin-orbit interaction and the Darwin term.\\
	\subsection{Relativistic Correction for the Kinetic Energy}
	The relativistic correction for kinetic energy is given by the expansion of $E=c\sqrt{p^2+m^2c^2}$. Expanding, we get
	\begin{equation}
		E\simeq mc^2+\frac{\opr{p}^2}{2m}-\frac{\opr{p}^4}{8m^3c^2}
		\label{eq:energypowerseries}
	\end{equation}
	We have our non perturbed Hamiltonian $\opr{\ham}_0$ with our hydrogenoid potential summed to the correction, as follows
	\begin{equation}
		\begin{aligned}
			\opr{\ham}&=\frac{\opr{p}^2}{2m}-\frac{Ze^2}{4\pi\epsilon_0r}+W_p\\
			W_p&=-\frac{\opr{p}^4}{8m^3c^2}
		\end{aligned}
		\label{eq:hampluspert}
	\end{equation}
	We already know that $\opr{p}^2=2m\left( \opr{\ham}_0-V \right)$, hence $\opr{p}^4=4m^2\left( \opr{\ham}_0-V \right)^2$, and
	\begin{equation*}
		W_p=-\frac{1}{2mc^2}\left( \opr{\ham}_0-V \right)^2
	\end{equation*}
	Surprisingly, this perturbation is diagonal in the basis of the Hydrogen atom, hence we find ourselves in need of calculating only some expectation values, as follows
	\begin{equation}
		\expval{W_p}=-\frac{1}{2mc^2}\expval{\opr{\ham}_0^2+V^2-\comm{\opr{\ham}_0}{V}}
		\label{eq:evaluationkineticcorr}
	\end{equation}
	Which gives
	\begin{equation}
		\expval{W_p}=-\frac{1}{2mc^2}\left( E_n^2+\frac{Z^2e^4}{(4\pi\epsilon_0)^2}\expval{\frac{1}{r^2}}+2E_n\frac{Ze^2}{4\pi\epsilon_0}\expval{\frac{1}{r}} \right)
		\label{eq:completecalc}
	\end{equation}
	The expectation values are easy to calculate and their explicit calculation is given in appendix \ref{app:E}. For our specific case, we get that
	\begin{equation}
		\begin{aligned}
			\expval{\frac{1}{r}}&=\frac{Z}{a_0n^2}\\
			\expval{\frac{1}{r^2}}&=\frac{Z^2}{a_0^2n^3\left( l+\frac{1}{2} \right)}\\
			\expval{\frac{1}{r^3}}&=\frac{Z^3}{a_0^3n^3l\left( l+\frac{1}{2} \right)\left( l+1 \right)}\quad l>0
		\end{aligned}
		\label{eq:expvalrminusk}
	\end{equation}
	Putting it all into our matrix elements of the perturbation, and remembering that $E_n=-E_1/n^2$, with $E_1=-Z^2/n^2\alpha^2mc^2$, with $\alpha=e^2/\hbar c$ is the fine structure constant, we get
	\begin{equation}
		\expval{W_p}=-\frac{1}{2mc^2}\left( E_n^2+\frac{Z^4e^4}{a_0^2n^3(4\pi\epsilon_0)^2(l+1/2)}+2E_n\frac{Z^2e^2}{4\pi\epsilon_0a_0n^2} \right)
		\label{eq:pertrelkinen}
	\end{equation}
	Substituting for $E_n$, we get, finally
	\begin{equation}
		\expval{W_p}=-E_n\frac{Z^2\alpha^2}{2n^2}\left( \frac{3}{4}-\frac{n}{l+1/2} \right)
		\label{eq:pertrelkinencomplete}
	\end{equation}
	For a Hydrogen atom we have $Z=1$, and the correction is of order $\alpha^2E_n$. Counting that $\alpha=137^{-1}$ the perturbation is small enough to be treated as such.\\
	\subsection{Darwin Term}
	The Darwin term $\left( \frac{\hbar^2}{8m^2c^2}\nabla^2V(r) \right)$, for Hydrogenoid atoms, becomes
	\begin{equation}
		\opr{W}_D=\frac{\hbar^2Ze^2}{8\epsilon_0m^2c^2}\delta(r_i)
		\label{eq:darwinterm}
	\end{equation}
	Where we used that $\nabla^2(1/r)=-4\pi\delta(r)$\\
	The matrix elements of $\opr{W}_D$ will be nonzero only for $l=0$, and will take the following form
	\begin{equation}
		\expval{W_D}=\frac{\hbar^2Ze^2}{8\epsilon_0m^2c^2}\abs{\phi_{nlm}(0)}^2
		\label{eq:darwinmatrixelements}
	\end{equation}
	In general $\abs{\phi_{n00}(0)}^2=(4\pi)^{-1}\abs{R_{n0}}^2$, where
	\begin{equation}
		R_{nl}(\rho)=\frac{1}{(2l+1)!}\sqrt{\left( \frac{2Z}{na_0} \right)^3\frac{(n+l)!}{2n(n-l-1)!}}e^{\frac{\rho}{2}}r^lF(l+1-n,2l+2,\rho)
		\label{eq:rnlgeneral}
	\end{equation}
	With $\rho=2Zr/na_0$. We have that $\abs{R_{n0}(0)}^2=\frac{4Z^3}{n^3a_0^3}$, hence
	\begin{equation}
		\abs{\phi_{n00}(0)}^2=\frac{Z^3}{\pi a_0^3n^3}
		\label{eq:phin00}
	\end{equation}
	For which, finally we get
	\begin{equation}
		\expval{W_D}=\frac{\hbar^2Z^4e^2}{8\pi\epsilon_0a_0^3m^2c^2n^3}=\frac{\hbar^2}{2m^2c^2}\frac{Z^2}{n^3a_0^2}\alpha^2mc^2=-E_n\frac{\hbar^2Z^2}{na_0m^2c^2}=-E_n\frac{Z^2\alpha^2}{n}
		\label{eq:darwintermpert}
	\end{equation}
	\subsection{Spin Orbit Coupling}
	The last term to evaluate, is the Spin-Orbit coupling of angular momentums, which gives the following perturbation
	\begin{equation}
		\opr{W}_{SO}=\frac{1}{2m^2c^2}\frac{1}{r}\derivative{V}{r}\vecopr{L}\cdot\vecopr{S}
		\label{eq:spinorbitpert}
	\end{equation}
	Explicitly
	\begin{equation}
		\opr{W}_{SO}=\frac{Ze^2}{8\pi\epsilon_0m^2c^2r^3}\vecopr{L}\cdot\vecopr{S}
		\label{eq:spinorbitpertexp}
	\end{equation}
	Its matrix elements will be, then
	\begin{equation}
		\expval{W_{SO}}=\frac{Ze^2}{8\pi\epsilon_0m^2c^2}\expval{\frac{1}{r^3}}\expval{\vecopr{L}\cdot\vecopr{S}}
		\label{eq:matrixelementssoint}
	\end{equation}
	Adding angular momentums and passing to the basis $\ket{jlsm_j}$, we get that
	\begin{equation*}
		\vecopr{L}\cdot\vecopr{S}=\frac{1}{2}\left( \opr{J}^2-\opr{L}^2-\opr{S}^2 \right)
	\end{equation*}
	And remembering that
	\begin{equation*}
		\expval{\frac{1}{r^3}}=\frac{Z^3}{a_0^3n^3l(l+1/2)(l+1)},\quad l>0
	\end{equation*}
	We get
	\begin{equation}
		\expval{W_{SO}}=\frac{\hbar^2Z^4e^2}{16\pi\epsilon_0a_0^3n^3l(l+1/2)(l+1)m^2c^2}\left( j(j+1)-l(l+1)-s(s+1) \right)
		\label{eq:wsocomplete}
	\end{equation}
	In which, substituting in the following values
	\begin{equation*}
		\begin{aligned}
			E_n&=-\frac{Z^2\alpha^2mc^2}{2n^2}\\
			a_0&=\frac{\hbar}{mc\alpha}\\
			\frac{e^2}{4\pi\epsilon_0}&=\alpha\hbar c
		\end{aligned}
	\end{equation*}
	We have
	\begin{equation}
		\expval{W_{SO}}=-E_n\frac{Z^2\alpha^2}{2n^2}\frac{j(j+1)-l(l+1)-s(s+1)}{l(l+1/2)(l+1)}
		\label{eq:wsosubstituted}
	\end{equation}
	Which, dividing it in two cases, if we have $j=l+1/2$ or $j=l-1/2$, we get
	\begin{equation}
		\expval{W_{SO}}=-E_n\frac{Z^2\alpha^2}{2nl(l+1/2)(l+1)}\cdot\left.\begin{dcases}l&j=l+1/2\\-(l+1)&j=l-1/2\end{dcases}\right\}
		\label{eq:spinorbitcomplete}
	\end{equation}
	The total, and final, perturbation given by the relativistic approximation will not depend on $l$, and it's given by the following formula
	\begin{equation}
		E_{nj}=E_n\left( 1+\frac{Z^2\alpha^2}{n^2}\left( \frac{n}{j+1/2}-\frac{3}{4} \right) \right)
		\label{eq:finalpertrelcorr}
	\end{equation}
	\subsection{Fine Structure Splitting}
	After summing all these perturbations to our initial energy, we have that although the non-relativistic energy levels were $2n^2$ times degenerate, we have that in the Dirac theory (i.e. relativistic quantum mechanics), we have that the $n-$th level splits in $n$ different levels, each one with its own value of $j$. This splitting is commonly called \textit{fine structure splitting}, and these $n$ levels are called \textit{fine structure multiplets}. The dimensionless \textit{fine structure constant} $\alpha\simeq1/137$ controls the scale of this splitting. It's important to note how in Dirac theory, two states with the same quantum numbers $n,j$ but with $l=j\pm1/2$, have the same energy, where the solution still has $(-1)^l$ parity. Thus for each $j$ we have two series of $2j+1$ solutions with opposite parity, except for $j=n-1/2$, for which there is only a series of solutions with parity $(-1)^{n-1}$. The splitting we talked about in this paragraph is indicated as follows with spectroscopic notation.\\
	e.g. let's say that we have $n=3$, hence $l=0,1,2$ and $j=1/2,\ 3/2,\ 5/2$. Through the perturbed energy \eqref{eq:finalpertrelcorr} we will have that the $n=3$ level will split in $5$ levels, as follows
	\begin{equation*}
		3s_{1/2},3p_{1/2},3p_{3/2},3d_{3/2},3d_{5/2}
	\end{equation*}
	This splitting is accompanied by a further splitting, called \textit{hyperfine splitting}, and its contribution is called \textit{Lamb shift}
	\subsection{Fine Structure Dipole Transitions}
	We already seen how the only permitted transitions in the dipole approximation are those that have $\Delta l=\pm1$ and $\Delta s=0$, which implies that $\Delta j=0,\pm1$. Since $\vecopr{J}$ eigenstates are linear combinations of eigenstates of $\vecopr{S},\vecopr{L}$, we have that $\Delta j=0$ transitions are permitted.\\
	e.g., let's see how it works for states with $l=1$ and $s=1/2$. There will be 6 states, where 4 will have $j=3/2$ and 2 will have $j=1/2$. The six states in the $\ket{jm_j}$ basis can be written as a linear combination of states $\ket{lsm_lm_s}$. Starting with $m_j=3/2$ we have
	\begin{equation*}
		\ket{\frac{3}{2},\frac{3}{2}}=\ket{1,\frac{1}{2},1,\frac{1}{2}}
	\end{equation*}
	Operating with $\ladoprd{J}$ we have
	\begin{equation*}
		\ket{\frac{3}{2},\frac{1}{2}}=A\ket{1,\frac{1}{2},0,\frac{1}{2}}+B\ket{1,\frac{1}{2},1,-\frac{1}{2}}
	\end{equation*}
	Analoguously, starting from $m_j=-3/2$ and working our way up the eigenstate ladder with $\ladopru{J}$, we get
	\begin{equation*}
		\ket{\frac{3}{2},-\frac{1}{2}}=C\ket{1,\frac{1}{2},0,-\frac{1}{2}}+D\ket{1,\frac{1}{2},-1,\frac{1}{2}}
	\end{equation*}
	And finally, the last two states
	\begin{equation*}
		\begin{aligned}
			\ket{\frac{1}{2},-\frac{1}{2}}&=-\sqrt{\frac{2}{3}}\ket{1,\frac{1}{2},-1,\frac{1}{2}}+\sqrt{\frac{1}{\sqrt{3}}}\ket{1,\frac{1}{2},0,-\frac{1}{2}}\\
			\ket{\frac{1}{2},\frac{1}{2}}&=E\ket{1,\frac{1}{2},1,-\frac{1}{2}}+F\ket{1,\frac{1}{2},0,\frac{1}{2}}
		\end{aligned}
	\end{equation*}
	In this situation there are 7 possible transitions, but there will be only 5 visible lines, since $\Delta j=2$ is not permitted by the selection rules.\\
	This can be seen as follows: before the absorption, the total angular momentum is $j_i$, of the electron, summed with the photon spin $s_{\gamma}=1$, hence the total (initial) angular momentum will be $k_i$, where $\abs{j_i-1}<k_i<j_i+1$. After the transition, we must have $k_f=j_f$, and since angular momentum must be conserved, we have that $k_i=k_f$, hence $\abs{j_i-1}<j_f<j_i+1$, this happens analogously with emission transitions.
	Finally, the selection rules for dipole transitions in fine structure systems are $\Delta l=\pm1,\ \Delta s=0,\ \Delta j=0,\pm1$ where $j=0\to j=0$ is not permitted
	\section{Zeeman Effect}
	Getting back to our semiclassical EM Hamiltonian, we have that the time dependent perturbation can be written as
	\begin{equation*}
		-\frac{i\hbar e}{m}\vecopr{A}\cdot\nabla=\frac{e}{2m}\vec{B}\cdot\vecopr{L}
	\end{equation*}
	And the quadratic perturbation as follows
	\begin{equation*}
		\frac{e^2}{2m}\opr{A}^2=\frac{e^2}{8m}\left( B^2\opr{r}^2-(\vec{B}\cdot\vecopr{r})^2 \right)
	\end{equation*}
	Since in laboratories is rare to exceed $10$ T of magnetic field intensity, hence since $4a_0^2B/4\hbar\approx 10^{-6}B$ we suppose that the quadratic term is again negligible.\\
	We define a \textit{Magnetic dipole moment operator} as follows.\\
	\begin{equation}
		\vecopr{M}=-\frac{e}{2m}\vecopr{L}=-\frac{\mu_B}{\hbar}\vecopr{L}
		\label{eq:mmomopr}
	\end{equation}
	With $\mu_B=e\hbar/2m$ is \textit{Bohr's magneton}, which has the value of around $9.27408\cdot10^{24}\ \mathrm{J/T}$ or $\mathrm{Am^2}$. The interaction Hamiltonian with this field then is
	\begin{equation}
		\opr{\ham}_{m}=-\vecopr{M}\cdot\vec{B}
		\label{eq:interactionhammag}
	\end{equation}
	To this we have to add the intrinsic magnetic moment given by the electron, which has the following shape
	\begin{equation}
		\vecopr{M}_s=-\frac{g\mu_B}{\hbar}\vecopr{S}
		\label{eq:spinmmopr}
	\end{equation}
	Where $g=2$ is the \textit{gyromagnetic ratio} of the electron
	And the following Hamiltonian
	\begin{equation}
		\opr{\ham}_s=-\vecopr{M}_s\cdot\vec{B}
		\label{eq:interactionhamspin}
	\end{equation}
	Our final Hamiltonian that accounts for spin-orbit interactions and atom-magnetic field interaction is simply given via the Pauli equation, at which has been ``attached'' the spin-orbit coupling element. The Pauli equation will then be %ik gotta fix%
	\begin{equation}
		\bra{r}\opr{\ham}\ket{\psi}=\left( -\frac{\hbar^2}{2m}\nabla^2-\frac{Ze^2}{4\pi\epsilon_0r}+\xi(r)\vecopr{L}\cdot\vecopr{S}+\frac{\mu_B}{\hbar}\left( \vecopr{L}+2\vecopr{S} \right)\cdot\vec{B} \right)\psi(r)=E\psi(r)
		\label{eq:sopauliequation}
	\end{equation}
	Where, our $\psi(r)$ not only accounts for spin, but for spin-orbit coupling also.
	\subsection{Strong Fields}
	Strong fields are characterized for having a magnetic field intensity of $B>Z^4$ Tesla. In this situation the spin-orbit coupling term is negligible, and our equation, after applying a rotation and having $\vec{B}||\ver{z}$, we have
	\begin{equation}
		\left( -\frac{\hbar^2}{2m}\nabla^2-\frac{Ze^2}{4\pi\epsilon_0r} \right)\psi(r_i)=\left( E-\frac{\mu_BB_z}{\hbar}\left( \opr{L}_z+2\opr{S}_z \right) \right)\psi(r_i)
		\label{eq:strongfieldham}
	\end{equation}
	The perturbation is diagonal, and the shift will be
	\begin{equation}
		E_{nm_lm_s}=E_n+\mu_BB_z(m_l+2m_s)
		\label{eq:energyshiftmagfield}
	\end{equation}
	Where in this case $m_s=\pm1/2$. Since there is no spin orbit coupling in this case, we have that levels with $m_l=1,m_s=-1/2$ and $m_l=-1,m_s=1/2$ coincide.\\
	We already know from the selection rules that we must have $\Delta m_s=0$ and $\Delta m_l=0,\pm1$, thus splitting the transition $n\to n'$ into three components. The two components with $\Delta m_l=\pm1$ are called $\pi$ lines, and the remaining one, with $\Delta m_l=0$ is called the $\sigma$ line.\\
	These $\pi$ transitions have the following frequencies
	\begin{equation}
		\omega^{\pm}_{n'n}=\omega_{n'n}\pm\omega_L
		\label{eq:pitranslines}
	\end{equation}
	Where $\omega_L=\mu_BB_z/\hbar$ is the \textit{Larmor frequency}. This effect is known as \textit{normal Zeeman effect}, and the $\pi$ and $\sigma$ lines of this effect are said to be \textit{Lorentz triplets}. They are also observed in atoms for which $S=0$, hence where spin-orbit coupling is absent.
	\subsection{Paschen-Back Effect}
	At field strengths for which spin orbit coupling is appreciable but still small with respect to the field intensity $B$, we have that first order perturbation theory can be applied in order to calculate the energy shift.\\
	We have then
	\begin{equation}
		\Delta E=\int_{0}^{\infty}r^2R_{nl}^2\xi(r)\diff{r}\bra{l\frac{1}{2}m_lm_s}\vecopr{L}\cdot\vecopr{S}\ket{l\frac{1}{2}m_lm_s}=\lambda_{nl}m_lm_s\quad l\ne0
		\label{eq:paschenbackshiftfirstorder}
	\end{equation}
	We have that<++>
	\begin{equation}
		\lambda_{nl}=-E_n\frac{Z^2\alpha^2}{n}\frac{1}{l(l+1/2)(l+1)}
		\label{eq:lambdapb}
	\end{equation}
	Which removes the degeneracy in $l$. The energy difference between two different levels, when $m_s=m_s'$ ($\Delta m_s=0$) is then the following
	\begin{equation}
		\Delta E=E_{n'}-E_n+E_{z}+E_{pb}=E_{n'}-E_n+\mu_BB_z(m_l'-m_l)+(\lambda_{n'l'}m_l'-\lambda_{nl}m_l)m_s
		\label{eq:paschenbackshift}
	\end{equation}
	\subsection{Anomalous Zeeman Effect}
	For weak magnetic fields, we have what's usually called the \textit{anomalous Zeeman effect}. In this case, the spin-orbit coupling is the dominant term, and our unperturbed Hamiltonian takes the following shape
	\begin{equation}
		\opr{\ham}_0=-\frac{\hbar^2}{2m}\nabla^2-\frac{Ze^2}{4\pi\epsilon_0r}+\xi(r)\vecopr{L}\cdot\vecopr{S}
		\label{eq:spinorbitham}
	\end{equation}
	This Hamiltonian has exact wavefunctions, which are given via what's known as \textit{Tensor Spherical Harmonics} (see appendix \ref{app:tsh}). Our unperturbed wavefunction will then be
	\begin{equation}
		\psi_{nlsjm_j}(r,\theta,\phi)=R_{nl}(r)\mc{Y}^{jm_j}_{ls}(\theta,\phi)=C_{jm_j}^{lsm_lm_s}R_{nl}(r)Y^{m_l}_l(\theta,\phi)\chi_{s,m_s}
		\label{eq:wavefunctionspinorbit}
	\end{equation}
	Or, in Dirac notation
	\begin{equation}
		\ket{nl}\otimes\ket{ls}\otimes\ket{jm_j}=\bra{lsm_lm_s}\left(\ket{jm_j}\otimes\ket{nl}\otimes\ket{lm_l}\otimes\ket{sm_s}\right)
		\label{eq:spinorbitautoket}
	\end{equation}
	Taking the magnetic field parallel to the $z$ axis, we have that the perturbation needed to be evaluated is
	\begin{equation}
		\opr{\ham}'=\frac{\mu_B}{\hbar}\left( \opr{J}_z+\opr{S}_z \right)B_z
		\label{eq:perturbationweakfield}
	\end{equation}
	And, the perturbation on the energy levels is
	\begin{equation}
		\Delta E=\mu_Bm_jB_z+\frac{\mu_B}{\hbar}B_z\sum_{m_s}\int\cc{\mc{Y}}^{jm}_{l,1/2}\opr{S}_z\mc{Y}^{jm}_{l,1/2}\diff{\Omega}
		\label{eq:energyshiftweakfield}
	\end{equation}
	From the properties of tensor spherical harmonics, we have that the integral is of easy computation, and gives the following result
	\begin{equation}
		\int\cc{\mc{Y}}^{l\pm1/2,m_j}_{l,1/2}\opr{S}_z\mc{Y}^{l\pm1/2,m_j}_{l,1/2}\diff{\Omega}=\pm\frac{\hbar m_j}{2l+1}
		\label{eq:integralcomputationweakf}
	\end{equation}
	Which gives our searched energy shift
	\begin{equation}
		\Delta E=\mu_Bm_jB_z+\frac{\mu_BB_z}{2l+1}\sum_{m_s}m_j
		\label{eq:anomalouszeemanshift}
	\end{equation}
	Finally, getting back to our Hydrogenoid atoms, we now can write the total perturbation of the energy levels for an interaction with a constant magnetic field as follows
	\begin{equation}
		E_{njm_j}=E_n+\Delta E_{nj}+\Delta E_{m_j}
		\label{eq:totalperthyatoms}
	\end{equation}
	Where $E_n$ is the unperturbed energy, $\Delta E_{nj}$ is the fine structure correction and $\Delta E_{m_j}$ is the weak field correction
	\section{Stark Effect}
	The splitting of energy levels given by static electric fields is called \textit{Stark effect}. We assume that the electric field is perpendicular to our $z$ axis, and that the field strength is much larger than Spin-Orbit coupling.\\
	The perturbation acting on our Hydrogenic Hamiltonian is
	\begin{equation}
		\opr{\ham}'=eEz
		\label{eq:starkpert}
	\end{equation}
	\subsection{Linear Stark Effect}
	In order to evaluate Stark shifts with perturbations at first order, we start by calculating the perturbation given to the fundamental state.\\
	Its first order correction is
	\begin{equation}
		E^{(1)}_{100}=eE\bra{100}\opr{z}\ket{100}
		\label{eq:starkgs}
	\end{equation}
	We already see that $E^{(1)}_{100}=0$, since the integral is null, due to it being the product of an even function ($\bra{x_i}\ket{nlm}$) with an odd one ($z$). We therefore check for excited states, starting at $n=2$, for which we have a fourfold degeneration given by $l$, with energy $E_{2lm}=-mc^2\alpha^2/8$.\\
	In this case we have that $\bra{nlm}\opr{z}\ket{n'l'm'}$ doesn't vanish, if and only if $\Delta m=0$ and $\Delta l=\pm1$, hence, the permitted transitions are only those between $2s$ and $2p$ states, for which our perturbation (which is real, hence Hermitian) gives the following result
	\begin{equation}
		E^{(1)}_{2s\to2p}=eE\int\cc{\psi}_{210}(x_i)z\psi_{200}(x_i)\diff[3]{x_i}
		\label{eq:energyshiftstarkeff}
	\end{equation}
	This last integral gives us that $E^{(1)}=\pm3eEa_0/Z$, for which we have two different energy levels, described by the following states
	\begin{equation}
		\begin{aligned}
			\ket{1}&=\frac{1}{\sqrt{2}}\left( \ket{200}+\ket{210} \right)\\
			\ket{2}&=\frac{1}{\sqrt{2}}\left( \ket{200}-\ket{210} \right)
		\end{aligned}
		\label{eq:eigenstateslinearstark}
	\end{equation}
%	\section{Lamb Shift}
	\section{Hyperfine Structure and Isotope Shifts}
	Hyperfine structure of energy levels appears when nucleii aren't considered anymore point charges with infinite mass. The name, \textit{hyperfine} has been given to this structure, since the deviation of the energy levels due to this additional perturbation is much smaller than that of the fine structure shifts.\\
	In general we can have two kinds of hyperfine effects: \textit{isotope shifts}, that slightly deviate degenerate energy levels without splitting them, and proper hyperfine effects that break the degeneration on the levels and splits them.\\
	\subsection{Magnetic Dipole Hyperfine Structure}
	As for the electrons, we can define a \textit{nuclear} spin, indicated with $\opr{I}$, which obey the same spin algebra of ordinary spin. As with electrons, the eigenvalues for nuclear spin can be half-integer, if the sum of the spins of the nucleons is fermionic, or integer, if the sum of the spins of the nucleons is bosonic. We denote the eigenvalues of $\opr{I}^2$ as $\hbar^2I(I+1)$ and of $\opr{I}_z$ as $\hbar M_I$.\\
	We can define, as for electrons, a \textit{Nuclear Magnetic Dipole Moment}, $\opr{M}_I$ as follows
	\begin{equation}
		\opr{M}_I=\frac{g_I\mu_N}{\hbar}\opr{I}
		\label{eq:nucleardipolemoment}
	\end{equation}
	Where $g_I$ is the \textit{Landé factor}, and $\mu_N$ is the \textit{Nuclear magneton}, defined as
	\begin{equation}
		\mu_N=\frac{e\hbar}{2m_p}=\frac{m_e}{m_p}\mu_B
		\label{eq:nuclearmagneton}
	\end{equation}
	We now proceed to write our perturbed Hamiltonian as follows
	\begin{equation}
		\opr{\ham}=\opr{\ham}_0+\opr{\ham}_{ND}
		\label{eq:pertrel}
	\end{equation}
	Since we have already solved for the fine structure Hamiltonian, it is included into the unperturbed Hamiltonian together with the Coulomb interaction.\\
	At the zeroth order we have that the wavefunctions of $\opr{\ham}$ are separable in electronic and nuclear variables, and are eigenfunctions of $\opr{J}^2,\opr{J}_z,\opr{I}^2,\opr{I}_z$. These wavefunction are $(2j+1)(2I+1)$ degenerate in $m_j,M_I$.\\
	Examining the perturbation $\opr{\ham}_{ND}$, we have that it will couple with both $\vecopr{L}$ and $\vecopr{S}$ of the electrons, hence splitting the Hamiltonian in a sum. Since this point dipole is located at the origin, we have that
	\begin{equation}
		\opr{\ham}_{NL}=-\frac{i\hbar e}{m}\vecopr{A}\cdot\nabla=\frac{\mu_0}{2\pi\hbar^2r^3}g_I\mu_B\mu_N\vecopr{L}\cdot\vecopr{I}
	\end{equation}
	For the spin-spin coupling Hamiltonian we have that, in terms of magnetic fields we can write the perturbation term as
	\begin{equation}
		\opr{\ham}_{SS}=-\vecopr{M}_s\cdot\vec{B}
		\label{eq:spinspinham}
	\end{equation}
	Where
	\begin{equation}
		\vec{B}=-\frac{\mu_0}{4\pi}\left( \vecopr{M}_N\nabla^2\frac{1}{r}-\nabla(\vecopr{M}_N\cdot\nabla)\frac{1}{r} \right)
		\label{eq:magneticdipolenucleus}
	\end{equation}
	Hence
	\begin{equation}
		\opr{\ham}_{SS}=-\frac{2\mu_0}{4\hbar^2\pi}g_I\mu_B\mu_N\left( \vecopr{S}\cdot\vecopr{I}\nabla^2\frac{1}{r}-(\vecopr{S}\cdot\nabla)(\vecopr{I}\cdot\nabla)\frac{1}{r} \right)
		\label{eq:spinspincoupling}
	\end{equation}
	At $r=0$ this perturbation will act only on s-states, and for $r\ne0$ we get
	\begin{equation*}
		\opr{\ham}_{SS}=-\frac{\mu_0}{4\pi r^3}\left( \vecopr{M}_s\cdot\vecopr{M}_N-\frac{3}{r^2}\left( (\vecopr{M}_s\cdot\vecopr{r})(\vecopr{M}_N\cdot\vecopr{r}) \right) \right)
	\end{equation*}
	Summing the two terms, we have that $\opr{\ham}_{ND}=\opr{\ham}_{NL}+\opr{\ham}_{SS}$, and we finally have
	\begin{equation}
		\opr{\ham}_{ND}=\frac{\mu_0}{2\hbar^2\pi}g_I\mu_B\mu_N\frac{1}{r^3}\left( \vecopr{L}\cdot\vecopr{I}-\vecopr{S}\cdot\vecopr{I}+3\frac{(\vecopr{S}\cdot\opr{r})(\vecopr{I}\cdot\opr{r})}{r^2} \right)\ r\ne0
		\label{eq:nucleardipole}
	\end{equation}
	Simplifying everything, we have, at $r=0$
	\begin{equation}
		\opr{\ham}_{ND}=-\frac{2\mu_0}{3\pi}\vecopr{M_s}\cdot\vecopr{M_N}\delta(\opr{r})\ r=0
		\label{eq:centernucleardipole}
	\end{equation}
	This last expression is called \textit{Fermi contact interaction}. Another way to write \eqref{eq:nucleardipole} is to define the new operator
	\begin{equation}
		\vecopr{G}=\vecopr{L}-\vecopr{S}+3\frac{(\vecopr{S}\cdot\vecopr{r})\vecopr{r}}{r^2}
		\label{eq:gangularmomnuc}
	\end{equation}
	And \eqref{eq:nucleardipole} becomes
	\begin{equation}
		\opr{\ham}_{ND}=\frac{\mu_0}{2\hbar^2\pi}g_I\mu_B\mu_N\frac{1}{r^3}\vecopr{G}\cdot\vecopr{I}
		\label{eq:nucleardipoleG}
	\end{equation}
	Before proceeding in the calculation of the energy shifts, we define a \textit{total angular momentum} for the whole system as $\vecopr{F}=\vecopr{I}+\vecopr{J}$, and the energy shift will be given in the total angular momentum basis as follows
	\begin{equation}
		\Delta E=\frac{\mu_0}{2\hbar^2\pi}g_I\mu_B\mu_N\bra{lsjIFM_F}\frac{1}{r^3}\vecopr{G}\cdot\vecopr{I}\ket{lsjIFM_F}
		\label{eq:energyshiftnucmom}
	\end{equation}
	We can write $\vecopr{G}\cdot\vecopr{I}$ as $(\vecopr{G}\cdot\vecopr{J})(\vecopr{I}\cdot\vecopr{J})/\hbar^2j(j+1)$, and knowing that $\vecopr{I}\cdot\vecopr{J}=1/2\left( \opr{F}^2-\opr{I}^2-\opr{J}^2 \right)$ we have the total shift as
	\begin{equation}
		\Delta E=\frac{C}{2}\left( F(F+1)-I(I+1)-j(j+1) \right)
		\label{eq:totalenergyshiftmnd}
	\end{equation}
	Where
	\begin{equation}
		C=\frac{\mu_0}{2\hbar^2j(j+1)\pi}g_I\mu_B\mu_N\expval{\frac{1}{r^3}\vecopr{G}\cdot\vecopr{J}}
		\label{eq:constantmdshift}
	\end{equation}
	Noting that $\vecopr{G}\cdot\vecopr{J}=\opr{L}^2$, we finally get our solution, having put $a_{\mu}$ as $a_0m/\mu$, with $\mu$ the reduced mass of the electron and the nucleus
	\begin{equation}
		\Delta E_{ND}=\frac{\mu_0}{4\pi}g_I\mu_B\mu_N\frac{l(l+1)}{j(j+1)}\frac{Z^3}{a_{\mu}^3n^3l(l+1/2)(l+1)}\left( F(F+1)-I(I+1)-j(j+1) \right)\ l\ne0
		\label{eq:finalsolenuclearmom}
	\end{equation}
	For $l=0$ we instead get
	\begin{equation}
		\Delta E_{ND}=\frac{2\mu_0}{3\pi}g_I\mu_B\mu_N\frac{Z^3}{a_{\mu}^3n^3}\left( F(F+1)-I(I+1)-s(s+1) \right)
		\label{eq:l0nuclearmom}
	\end{equation}
	In atomic units, the final result will then be
	\begin{equation}
		\Delta E_{hyp}=\frac{m_e}{m_p}\left( \frac{\mu}{m_e} \right)^3\frac{2g_IZ^3\alpha^2}{n^3(j+1)(2l+1)}\cdot\left.\begin{dcases}I+\frac{1}{2}&j\le I\\
		\frac{I(j+1/2)}{j}&j\ge I\end{dcases}\right\}
		\label{eq:hyperfinetrans}
	\end{equation}
	We see that the total angular momentum quantum number behaves as $\vecopr{J}$, hence $\Delta F=0,\pm1$ are the only permitted transitions, with $0\to0$ being excluded
\end{document}
