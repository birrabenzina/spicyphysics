\documentclass[../qm.tex]{subfiles}
\begin{document}
	We start by directly writing the Schrödinger equation for a two electron atom in atomic units, where
	\begin{equation*}
		\begin{aligned}
			\hbar&=1\\
			k_e&=\frac{1}{4\pi\epsilon_0}=1\\
			e&=1
		\end{aligned}
	\end{equation*}
	We immediately have
	\begin{equation}
		\opr{\ham}=-\frac{1}{2}\nabla_1^2-\frac{1}{2}\nabla_2^2-\frac{Z}{r_1}-\frac{Z}{r_2}+\frac{1}{r_{12}}
		\label{eq:twoelectronham}
	\end{equation}
	In this case we have that the wavefunction is simmetric to spatial exchange of the two electrons (also called para-wavefunction, similarly spatially antisymmetric wavefunction are called ortho-wavefunctions).\\
	We also must impose the Pauli exclusion principle, by taking into account the electrons' spin. We end up having our wavefunction as
	\begin{equation}
		\Psi(q_1,q_2)=\psi(x_i^1,x_i^2)\chi_{1/2,m_s}(1,2)
		\label{eq:wavefunctiontwoeminus}
	\end{equation}
	As we know already, the basis spinor wavefunction for a system of two electrons can be either antisymmetric (singlet) or symmetric (triplet), and hence can take the following shapes
	\begin{equation*}
		\begin{aligned}
			\ket{00}&=\frac{1}{\sqrt{2}}\left( \ket{\up}\ket{\down}-\ket{\down}\ket{\up} \right)\\
			\ket{11}&=\ket{\up}\ket{\up}\\
			\ket{10}&=\frac{1}{\sqrt{2}}\left( \ket{\up}\ket{\down}+\ket{\down}\ket{\up} \right)\\
			\ket{1\down1}&=\ket{\down}\ket{\down}
		\end{aligned}
	\end{equation*}
	Due to Pauli's exclusion principle, we have that the final wavefunction must be completely antisymmetric, hence, if we have a para wavefunction $\psi_+$, the final solution will take the shape
	\begin{equation}
		\Psi(q_1,q_2)=\psi_+(x_i^1,x_i^2)\chi_{00}
		\label{eq:parawave}
	\end{equation}
	Or analoguously, if we have an ortho wavefunction
	\begin{equation}
		\Psi(q_1,q_2)=\psi_-(x_i^1,x_i^2)\cdot\left.\begin{dcases}\chi_{11}\\\chi_{10}\\\chi_{1-1}\end{dcases}\right\}
		\label{eq:orthowave}
	\end{equation}
	Written this, one may immediately ask what's then the scheme of energy levels? As an example we take an Helium atom. We have that $Z=2<<40$, hence we have a set of levels of an almost independent levels, one of ortho-triplet states and one of para-singlet states. The lowest energy levels, are divided then between singlets ($S=0$) and triplets ($S=1$), where $S$ is the total spin.\\
	Let $L$ be the sum of the eigenvalues of the square angular momentum of both electrons and $M_L$ the sum of the eigenvalues of the projection. We can introduce the \textit{spectroscopic terms} as a new notation for atomic energy levels. They're written as follows
	\begin{equation}
		^{2S+1}L_{M_J}
		\label{eq:atomictermsymboldef}
	\end{equation}
	The term $L$, analogously to the term $l$ for particles, takes the ``values'' $S,P,D,F,G,H,\cdots$, as $l$, which takes values $s,p,d,f,g,h,\cdots$. In addition, on the top left of the term there is indicated the \textit{multiplicity of the state}, which indicates whether it's a singlet or a triplet state, in terms of total spin. On the bottom right there are indicated the possible $M_J$ values of the considered system.
	\section{Independent Particle Model}
	In order to get a first approach to two-electron atoms, we need to develop an approximate theory, in which the $e^--e^-$ interaction is taken as a perturbation on the system
	\begin{equation}
		\begin{aligned}
			\opr{\ham}&=-\frac{1}{2}\nabla^2_1-\frac{Z}{r_1}-\frac{1}{2}\nabla^2_2-\frac{Z}{r_2}+\opr{\ham}'\\
			\opr{\ham}'&=\frac{1}{r_{12}}\\
			\opr{\ham}&=\opr{h}_1+\opr{h}_2+\opr{\ham}'=\opr{\ham}_0+\opr{\ham}'
		\end{aligned}
		\label{eq:indepparticles}
	\end{equation}
	In this case we have that the eigenvalues of the single electron Hamiltonian ($\opr{h}_i$) are known to be the following
	\begin{equation*}
		\opr{h}_i\psi_{nlm_i}(x^i_j)=-\frac{Z^2}{2n_i^2}\psi_{nlm_i}(x^i_j)\quad i=1,2;\ j=1,2,3
	\end{equation*}
	Hence, in general, we then have the following solution for the unperturbed Hamiltonian
	\begin{equation}
		\opr{\ham}_0\psi^0(x^1_i,x^2_i)=-\frac{Z^2}{2}\left( \frac{1}{n_1^2}+\frac{1}{n_2^2} \right)\psi_{n_1l_1m_1}(x_i^1)\psi_{n_2l_2m_2}(x_i^2)
		\label{eq:unpertsol2el}
	\end{equation}
	This solution adds up a new exchange degeneracy of the state. We already know tho that the final wavefunction must be the symmetrization or antisymmetrization of the two single electron wavefunctions. In Dirac notation, using $\ket{1,2}$ for the full wavefunction and $\ket{i}$ for the single electron wavefunction, we must then have
	\begin{equation}
		\ket{1,2}_{\pm}=\frac{1}{\sqrt{2}}\left( \ket{1}\ket{2}\pm\ket{2}\ket{1} \right)
		\label{eq:twoelectronatomwf}
	\end{equation}
	Therefore, we can take the antisymmetric wavefunctions ($\psi^0_-$) as an approximation for ortho states and ($\psi^0_+$) for para states.\\
	This consideration lets the ortho wavefunction vanish for the ground state of the system, in accord with the Pauli exclusion principle, hence there can exist a single para-singlet state as ground state, for which, the associated wavefunction is the following
	\begin{equation}
		\bra{1,2}\ket{1s,1s}=\psi_{GS}^0=\psi_{1s}(x_i^1)\psi_{1s}(x_i^2)=\frac{Z^3}{\pi}e^{-Z(r_1+r_2)}
		\label{eq:independentgs}
	\end{equation}
	This result, gives then for Helium ($Z=2$) $E_{GS}^0=-Z^2=-4$ a.u. ($-108.8$ eV), which corresponds to a ionization potential of $I_p=Z^2/2=2$ a.u ($54.4$ eV). The experimental values are not in accord with these results, which are $E_{GS}^{exp}=-2.90$ a.u. ($-79.0$ eV) and a ionization potential $I_p^{exp}=0.90$ a.u. ($24.6$ eV).\\
	This final result is quite far from the experimental result, since we neglected the $e^--e^-$ interaction.
	\subsection{Central Field Approximation}
	We now take again our equation \eqref{eq:indepparticles} and we take a new unperturbed Hamiltonian, which is the sum of the two single particle hamiltonians
	\begin{equation}
		\opr{h}_i'=-\frac{1}{2}\nabla_i^2+V(r_i)
		\label{eq:newsingleparthamind}
	\end{equation}
	And we modify the perturbation, getting this new formula
	\begin{equation}
		\opr{\ham}'=\frac{1}{r_{12}}-\frac{Z}{r_1}-V(r_1)-\frac{Z}{r_2}-V(r_2)
		\label{eq:newpertindel}
	\end{equation}
	In this new representation of the problem, we need to choose a central potential $V(r_i)$, for which we have that $\opr{\ham}'$ is small enough. We choose his potential to be the following
	\begin{equation*}
		V(r)=-\frac{Z-S}{r}=-\frac{Z_e}{r}
	\end{equation*}
	Where $S$ is some constant, called \textit{screening constant}. What we indicated with $Z_e$ is commonly considered as an \textit{effective charge}.\\
	Since the new potential is a Coulomb potential, we already know the ground state wavefunction and the energy levels, since we might simply insert the substitution $Z\to Z_e$ in what we have already found.\\
	Writing $u_{nlm}$ as the single particle equation, we will have that the single-particle energy levels will depend directly on $l$, breaking the degeneracy on that quantum number. Our complete wavefunction for discrete excited states will then be the following
	\begin{equation}
		\psi_{\pm}'(x_i^1,x_i^2)=\frac{1}{\sqrt{2}}\left( u_{100}(x_i^1)u_{nlm}(x_i^2)\pm u_{nlm}(x_i^1)u_{100}(x_i^2) \right)
		\label{eq:generalpsitwononinteminus}
	\end{equation}
	The total energy of the atom will then be, simply
	\begin{equation}
		E^0_{nl}=E_{1s}+E_{nl}
	\end{equation}
	Where $E_{nl}$ depends on the chosen potentia $V(r_i)$, and is $E_{nl}=-Z_{nl}^2/n^2$.\\
	In the special case of a completely screened electron we will have $S=1$, and the energy levels will then be described, in the non-interacting electron approximation as follows
	\begin{equation}
		E_n^0=-\frac{Z^2}{2}-\frac{(Z-1)^2}{2n^2}
		\label{eq:newlevels}
	\end{equation}
	\section{Ground State of Two Electron Atoms}
	\subsection{Perturbation Theory}
	We have already seen that the ground state eigenket of two electron atoms, in its most general expression can be written as follows
	\begin{equation}
		\ket{q_1,q_2}=\frac{1}{\sqrt{2}}\ket{1,2}\left( \ket{\up}_1\ket{\down}_2-\ket{\down}_1\ket{\up}_2 \right)
		\label{eq:2eminusatomgsstate}
	\end{equation}
	Using perturbation theory, we can already get a nice guess of the ground state correction, via the calculus of the matrix elements $\bra{1s,1s}\opr{\ham}'\ket{1s,1s}$, where $\opr{\ham}'=r_{12}^{-1}$. Writing explicitly the integral, we get
	\begin{equation}
		E_0^1=\int_{0}^{\infty}\abs{\psi_{1s}(r_1)}^2\frac{1}{r_{12}}\abs{\psi_{1s}(r_2)}^2\ddiff{r_1}{r_2}
		\label{eq:ptgspert}
	\end{equation}
	Using the following conversion, and the connection between Legendre polynomials and spherical harmonics, we have that, firstly
	\begin{equation*}
		\frac{1}{r_{12}}=\sum_{l=0}^{\infty}\frac{\min^l(r_1,r_2)}{\max^{l+1}(r_1,r_2)}P_l(\cos\theta)
	\end{equation*}
	Hence
	\begin{equation*}
		\frac{1}{r_{12}}=\sum_{l=0}^{\infty}\sum_{m=-l}^l\frac{4\pi}{2l+1}\frac{\min^l(r_1,r_2)}{\max^{r+1}(r_1,r_2)}\cc{Y}^m_l(\theta_1,\phi_1)Y^m_l(\theta_2,\phi_2)
	\end{equation*}
	Plugging it into the integral, and using the properties of spherical harmonics, we get
	\begin{equation*}
		E_0^1=\frac{Z^6}{\pi^2}\sum_{l=0}^{\infty}\sum_{m=-l}^l\frac{16\pi^2}{2l+1}\int_{0}^{\infty}r_1^2\diff{r_1}\int_{0}^{\infty}e^{-2Z(r_1+r_2)}\frac{r_m^l}{r_M^{l+1}}\kd{l0}\kd{m0}\diff{r_2}
	\end{equation*}
	Contracting the indices with the sums and the Kronecker deltas, we have our final integral
	\begin{equation}
		E_0^1=16Z^6\int_{0}^{\infty}e^{-2Zr_1}r_1^2\diff{r_1}\left( \frac{1}{r_1}\int_{0}^{r_1}e^{-2Zr_2}r_2^2\diff{r_2}+\int_{r_1}^{\infty}e^{-2Zr_2}r_2\diff{r_2} \right)
		\label{eq:energygshegspt}
	\end{equation}
	The integrals give the value $5/128 Z^5$, hence, we get our correction
	\begin{equation}
		E_0^1=\frac{5}{8}Z
		\label{eq:pthegscorr}
	\end{equation}
	The final approximate energy is then
	\begin{equation}
		E_0\approx E_0^0+E_0^1=-Z^2+\frac{5}{8}Z
		\label{eq:pthecorrtotal}
	\end{equation}
	\subsection{Variational Methods}
	Proceding instead using variational calculus, we set our $\ket{1s}$ wavefunction to be dependent from a parameter $Z_e$, which will be needed to minimize the following functional
	\begin{equation}
		E[\phi]=\frac{\bra{\phi}\opr{\ham}\ket{\phi}}{\braket{\phi}}
		\label{eq:phiwavefunction}
	\end{equation}
	Where we set $\ket{\phi}=\ket{1s}_1\otimes\ket{1s}_2$.\\
	Calculating and introducing here the quantum virial theorem, we get that
	\begin{equation}
		\bra{\phi}\frac{\opr{p}^2_1}{2m}\ket{\phi}=\bra{\phi}\frac{\opr{p}^2_2}{2m}\ket{\phi}=\bra{1s}\frac{\opr{p}^2}{2m}\ket{1s}=\frac{1}{2}Z_e^2
		\label{eq:expectedvalueheliumelectrons}
	\end{equation}
	We then also have
	\begin{equation}
		-\bra{\phi}\frac{Z}{r_1}+\frac{Z}{r_2}\ket{\phi}=-2ZZ_e
		\label{eq:potentialexpval}
	\end{equation}
	And, due to our previous calculations
	\begin{equation}
		\bra{\phi}\frac{1}{r_{12}}\ket{\phi}=\frac{5}{8}Z_e
		\label{eq:r12expval}
	\end{equation}
	The final result is our energy as a function of the parameter $Z_e$, which is
	\begin{equation}
		E(Z_e)=Z_e^2-2ZZ_e+\frac{5}{8}Z_e
		\label{eq:heliumenergyparameter}
	\end{equation}
	We search for an extremal of this function deriving with respect to the parameter, and we get
	\begin{equation}
		\begin{aligned}
			\pdv{E}{Z_e}&=2Z_e-2Z+\frac{5}{8}\\
			Z_e&=Z-\frac{5}{16}
		\end{aligned}
		\label{eq:parametervalue}
	\end{equation}
	Hence, since our energy was $E_0=-Z_e^2$, we get, in atomic units, that for Helioid atoms, the energy of the ground state will be approximated as follows
	\begin{equation}
		E_0^{GS}=-\left( Z-\frac{5}{16} \right)^2=-Z^2+\frac{5}{8}Z-\frac{25}{256}
		\label{eq:heliumgsfinal}
	\end{equation}
	Plugging in the values for Helium, we find that $E_0=2.84766\ \mathrm{au}$, a value that differs only by the $0.056\ \mathrm{au}$ from the experimental value.\\
	Another way of finding a more precise ground state energy has been found by Hylleraas, where another set of coordinates is chosen, and the trial wavefunction has multiple variational parameters. These are the following coordinates
	\begin{equation*}
		\begin{aligned}
			s&=r_1+r_2\\
			t&=r_1-r_2\\
			u&=r_{12}=\norm{\vec{r}_1-\vec{r}_2}
		\end{aligned}
	\end{equation*}
	The trial wavefunction is the following
	\begin{equation*}
		\phi(s,t,u)=e^{-ks}\sum_{l=0}^N\sum_{m=0}^N\sum_{n=0}^Nc_{l,2m,n}s^lt^{2m}u^m
	\end{equation*}
	Where $c_{l,2m,n}$ are linear variational parameters, and $k$ is a nonlinear variational parameter that behaves like the effective charge $Z_e$.\\
	Since the ground state must be a para state, this wavefunction must be an even function of $t$. Calculating for $N=5$, we have $6$ variational parameters, we have that the energy calculated is $-2.90324\ \mathrm{au}$, which is surprisingly close to the experimental value, differing only by $4.8\cdot10^{-4}\ \mathrm{au}$ ($0.013\ \mathrm{eV}$).
	\section{Excited States of Two Electron Atoms}
	We will now treat the excited states of Helioid atoms, with the same methods used for evaluating the ground states.
	\subsection{Perturbation Theory}
	Firstly, we apply perturbation theory to our system, formed by the sum $\opr{\ham}_0+\opr{\ham}'$, with $\opr{\ham}'=r_{12}^{-1}$.\\
	We already know that these states must either be parastates or orthostates, hence we will need our wavefunction to be either exchange-symmetric or exchange-antisymmetric. The energy perturbation will then depend on the sign of the wavefuction. We also know that $\comm{\opr{\ham}'}{\opr{L}_z}=0$, hence we can write the following result
	\begin{equation}
		E_{\pm}^1=\bra{\psi^0_{\pm}}\opr{\ham}'\ket{\psi^0_{\pm}}=J\pm K
		\label{eq:energyplusminusexcited}
	\end{equation}
	Where
	\begin{equation}
		\begin{aligned}
			J&=\int_{\mathbb{R}^3\times\mathbb{R}^3}\abs{\psi_{100}(\vec{r}_1)}^2\frac{1}{r_{12}}\abs{\psi_{nlm}(\vec{r}_2)}^2\ddiff[3]{r_1}{r_2}\\
			K&=\int_{\mathbb{R}^3\times\mathbb{R}^3}\cc{\psi_{100}}(\vec{r}_1)\cc{\psi_{nlm}}(\vec{r}_2)\frac{1}{r_{12}}\psi_{100}(\vec{r}_2)\psi_{nlm}(\vec{r}_1)\ddiff[3]{r_1}{r_2}
		\end{aligned}
		\label{eq:jkintegrals}
	\end{equation}
	Where $n\ge2$. The integral $J$ is called \textit{Coulomb or direct integral} and $K$ is called \textit{Exchange integral}.\\
	Expanding $r_{12}^{-1}$ in spherical harmonics, we get
	\begin{equation}
		\begin{aligned}
			J_{nl}&=\int_{0}^{\infty}R_{nl}(r_2)r_2^2\diff{r_2}\int_{0}^{\infty}R_{10}^2(r_1)\frac{r_1^2}{\max(r_1,r_2)}\diff{r_1}\\
			K_{nl}&=\frac{1}{2l+1}\int_{0}^{\infty}R_{10}(r_2)R_{nl}(r_2)r^2_2\diff{r_2}\int_{0}^{\infty}R_{10}(r_1)R_{nl}(r_1)\frac{\min^l(r_1,r_2)}{\max^{l+1}(r_1,r_2)}r_1^2\diff{r_1}
		\end{aligned}
		\label{eq:jkcompleteshexpansion}
	\end{equation}
	Hence, $E_{\pm}^1=E_{nl\pm}^1$, so the energy after this correcion directly depends on $n,l$.\\
	At first order, we can then write
	\begin{equation}
		E_{nl\pm}\approx E_{1n}^0+E_{nl\pm}^1=-\frac{Z^2}{2}\left( 1+\frac{1}{n^2} \right)+J_{nl}\pm K_{nl}
		\label{eq:energyshiftjkintegrals}
	\end{equation}
	In order to see how this perturbation acts, we can see that $J_{nl}$ must be always positive. Seeing that for $l=n-1$, $R_{n,n-1}$ has no nodes, hence $K_{n,n-1}>0$. It can also be seen that, in general $K_{nl}>0$, so, from $\eqref{eq:energyshiftjkintegrals}$ we see that an orthostate has an energy lower to the corresponding parastate.\\
	This can be seen introducing spin into our calculus, as follows.
	\begin{equation}
		E_{nl\pm}^1=J_{nl}-\frac{1}{2}\left( 1+\sigma_i^1\sigma^i_2 \right)K_{nl}
		\label{eq:spinenergyjk}
	\end{equation}
	Where $\sigma_i^j$ are the pauli matrices of the two electrons.\\
	\subsection{Variational Methods}
	Variational methods can be applied supposing that the higher-order wavefunction are orthogonal to the ground-state trial function and then calculate the variational integrals.\\
	This method is best applied on singular states.\\
	Starting with the $2^3S$ state of the atom, we have that, using variational parameters $Z_i,Z_o$ as the inner and outer effective charge, we can write the state as follows
	\begin{equation}
		\ket{2^3S}=N\left( \ket{1s}\ket{2s}-\ket{2s}\ket{1s} \right)
		\label{eq:tripletSvarstate}
	\end{equation}
	Where
	\begin{equation}
		\begin{aligned}
			u_{1s}(r)&=e^{-Z_ir}\\
			v_{2s}(r)&=\left( 1-\frac{Z_or}{2} \right)e^{-\frac{Z_or}{2}}
		\end{aligned}
		\label{eq:trialwavefunctionstripletS}
	\end{equation}
	Substituting into the variational integral and finding the minimum, we get $Z_i=2.01$ au and $Z_o=1.53$ au, which yeld $E_{2^3S}=-2.167$ au. Analogously, one can use Hylleraas wavefunction in order to get a more precise result.\\
	For $2^1P$ and $2^3P$ states, we use the following state kets
	\begin{equation}
		\begin{aligned}
			\ket{2^1P}&=N_+\left( \ket{1s}_1\ket{2pm}_2+\ket{2pm}_1\ket{2s}_2 \right)\\
			\ket{2^3P}&=N_-\left( \ket{1s}_1\ket{2pm}_2-\ket{2pm}_1\ket{2s}_2 \right)
		\end{aligned}
		\label{eq:2SPstateshe}
	\end{equation}
	With wavefunctions
	\begin{equation}
		\begin{aligned}
			u_{1s}(r)&=e^{-Z_ir}\\
			v_{2pm}(r)&=re^{\frac{Z_0r}{2}}Y^1_m(\ver{r})
		\end{aligned}
		\label{eq:wavefunctions2pstates}
	\end{equation}
	The variation gives, in atomic units
	\begin{equation*}
		\begin{aligned}
			Z_i^{2^1P}&=2.00\\
			Z_o^{2^1P}&=0.97\\
			E_{2^1P}&=-2.123\\
			Z_i^{2^3P}&=1.99\\
			Z_o^{2^3P}&=1.09\\
			E_{2^3P}&=-2.131
		\end{aligned}
	\end{equation*}
	The two theoretical values are around $3$ au from the experimental measurements.\\
\end{document}
