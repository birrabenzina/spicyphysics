\documentclass[../qm.tex]{subfiles}
\begin{document}
\section{A Small Intro to Some Wacky Units}
Since in particle physics usually we deal with particular calculations, it's preferable to avoid using the SI system of units, and instead pass to what I like to call, the system of \emph{God given units}, where the most common fundamental constants are taken as unitary, i.e. $\hbar=c=1$.\\
With this choice is also common to write energy in terms of $\mathrm{eV}$, i.e., using $c=1$ and $E=\gamma mc^2$, we have that
\begin{equation*}
	\left[ E \right]=\left[ m \right]=\mathrm{eV}
\end{equation*}
Using the dispersion relation we also get
\begin{equation*}
	E=p^2+m^2\implies{}\left[ p \right]=\mathrm{eV}
\end{equation*}
Therefore mass, energy and momentum are all expressed in $\mathrm{eV}$. Using that $1\ \mathrm{J}=(1.602)^{-1}\times 10^{19}\ \mathrm{eV}$ we get the conversion value
\begin{equation}
	1\ \mathrm{kg}=5.6\times 10^{35}\ \mathrm{eV}
	\label{eq:kgtoevconv}
\end{equation}
In these units we have that the mass of the electron $m_e$ and the mass of the proton $m_p$ are
\begin{equation}
	\begin{aligned}
		m_e&=9.109\times 10^{-31}\ \mathrm{kg}=0.511\ \mathrm{MeV}\\
		m_p&=1.673\times 10^{-27}\ \mathrm{kg}=938.3\ \mathrm{MeV}
	\end{aligned}
	\label{eq:massemassprot}
\end{equation}
The second consequence of taking $\hbar=c=1$ is that time can also be expressed in terms of $\mathrm{eV}$. In fact since $\left[ \hbar \right]=Js$ in the SI system, and $\hbar=1$ in the GGS\footnote{God Given System}, we have
\begin{equation*}
	\hbar=1.055\times10^{-34}\ \mathrm{Js}=6.583\times 10^{-22}\ \mathrm{MeVs}
\end{equation*}
Therefore
\begin{equation}
	1\ \mathrm{s}=\frac{1}{\hbar}\ \mathrm{MeV^{-1}}=1.519\times10^{21}\ \mathrm{MeV^{-1}}
	\label{eq:sinvmev}
\end{equation}
Combining both $\hbar c$ we have $\left[ \hbar c \right]=MeV m$, therefore we can think of expressing distances with this unit. Multiplying the constants we get
\begin{equation*}
	\hbar c=197.35\times10^{-15}\ \mathrm{MeV m}=197.35\ \mathrm{MeV fm}
\end{equation*}
This implies that
\begin{equation}
	1\ \mathrm{fm}=5.608\ \mathrm{GeV^{-1}}
	\label{eq:fmtogev}
\end{equation}
In these units is also quick to see that
\begin{equation}
	\alpha=\frac{e^2}{4\pi\epsilon_0\hbar c}\frac{1}{137}
	\label{eq:alphaconst}
\end{equation}
\section{Cross Section}
\subsection{Geometric Cross Section}
\subsection{Mean Free Path}
\subsection{Luminosity and Particle Accelerators}
\subsection{Differential Cross Sections}
\section{Cross Sections in Particle Physics}
\subsection{Rutherford Cross Section}
%check last
%\subsection{}<++>

\end{document}
